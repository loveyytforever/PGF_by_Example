\documentclass{standalone}
\usepackage{pgfplots}
\pgfplotsset{compat=newest}
\usepackage{pgfplotstable}
\usepackage{array}
\usepackage{colortbl}
\usepackage{booktabs}
\usepackage{eurosym}
\usepackage{amsmath}
\usepackage{pgfplotstable}
\usepackage{listings}
\usepackage{multirow}
\begin{document}
% An example how to use 
% \usepackage{multirow} and
% \usepackage{booktabs}:
\pgfplotstabletypeset[
	columns/Z/.style={
		column name={},
		assign cell content/.code={% use \multirow for Z column:
			\ifnum\pgfplotstablerow=0
				\pgfkeyssetvalue{/pgfplots/table/@cell content}
					{\multirow{4}{*}{##1}}%
			\else
				\pgfkeyssetvalue{/pgfplots/table/@cell content}{}%
			\fi
		},
	},
	% use \booktabs as well (compare examples above):
	every head row/.style={before row=\toprule,after row=\midrule},
	every last row/.style={after row=\bottomrule},
	row sep=\\,col sep=&,
	outfile=pgfplotstable.multirow.out,% write it to file
]{% here: inline data in tabular format:
	Z    & a & b \\
	data & 1 & 2 \\
	     & 3 & 4 \\
	     & 5 & 6 \\
	     & 7 & 8 \\
}
% ... and show the generated file:
\lstinputlisting[basicstyle=\footnotesize\ttfamily]{pgfplotstable.multirow.out}
\end{document}
